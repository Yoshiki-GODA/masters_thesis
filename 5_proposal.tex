\chapter {チャプター名}
\label{chap:提案手法}
\begin{comment}
    ストーリー:
    固定小数点演算は浮動小数点演算よりも計算時間が速いとされている.
    しかし,浮動小数点に比べて精度が低くなる恐れがある.
    本論文では,固定小数点と浮動小数点での数値計算の結果を比べ,固定小数点での演算も浮動小数点と同程度の精度で計算できることを示した.
    実験結果より固定小数点演算で浮動小数点より高速で,浮動小数点と同程度の計算を実現できるのではないかと考える.
\end{comment}
本論文で行う実験は,数値計算を固定小数点によって計算することである.
浮動小数点を用いて計算することの多い問題に対して固定小数点を用いて計算する方が計算結果の精度が高くなる場合があると考える.
次の章では,数値実験によって等しいbit数の固定小数点と浮動小数点を用いて実験を行った.


第\ref{chap:基礎知識1}章で述べたように,固定小数点と浮動小数点では表現できる数のうち隣り合う2数の間隔が異なる範囲がある.
そのため,固定小数点と浮動小数点では計算の過程で扱う数の範囲によって計算機の中で表現できる数の個数が異なる. %この文変だなぁ。
本論文では,ある数値型$\mathrm{T}$と計算機で扱う数の範囲$I \subset \R$を用いて,範囲$I$での数値型$\mathrm{T}$の密度${|I|}_{\mathrm{T}}$を以下のように定義する.
\begin{align}
    \label{eq:def_density}
    {|I|}_{\mathrm{T}}: \text{集合}\left\{x^{\ast} \ | \ x^{\ast} \in \mathbb{F}_{\mathrm{T}} \cap I \right\}\text{の要素の数.}
\end{align}
ただし,数値型$\mathrm{T}$で表せる最大の数を$M_{\mathrm{T}}$,最小の数を$m_{\mathrm{T}}$とし,
\begin{equation}
    I \subset (m_{\mathrm{T}}, M_{\mathrm{T}})
\end{equation}
とする.
例として,8bitの固定小数点(Q4f3)を考える.
このとき,
\begin{align}
    m_{\mathrm{Q}4\mathrm{f}3} &= -14.875, \\
    M_{\mathrm{Q}4\mathrm{f}3} &= 14.875
\end{align}
であり,$I$を
\begin{eqnarray}
    I = \left(-3,2\right)
\end{eqnarray}
とすると,${|\left(-3,2\right)|}_{\mathrm{Q}4\mathrm{f}3}$は,
\begin{align}
    &\left\{x^{\ast}\in \fixnumcustom{4}{3} \cap (-3,2)\right\} \\
    &= \left\{-2.875,-2.750,-2.625,\dots,1.625,1.750,1.875\right\}
\end{align}
となるため,
\begin{align}
    {|I|}_{\mathrm{Q}4\mathrm{f}3} = 39
\end{align}
となる.
このように定義した密度${|I|}_{\mathrm{T}}$を用いて,同じ範囲$J \in \R$に対して異なる数値型$T_1,T_2$で計算を行うことを考える.
ただし,
\begin{equation}
    J \subset (m_{T_1}, M_{T_1}) \cap (m_{T_2}, M_{T_2})
\end{equation}
とする.
また,
\begin{equation}
    \label{eq:density_compare}
    {|J|}_{T_1} < {|J|}_{T_2}
\end{equation}
と仮定する.

\begin{figure}[H]
    \centering
    \begin{tikzpicture}
        \node[text=red] (A-4) at (-3.5,0) {$\bigtriangleup$};
        \node[text=red] (A-3) at (-2.2,0) {$\bigtriangleup$};
        \node[text=red] (A-2) at (-2,0) {$\bigtriangleup$};
        \node[text=red] (A-1) at (-1,0) {$\bigtriangleup$};
        \node[text=red] (A0) at (0,0) {$\bigtriangleup$}; 
        \node[text=red] (A1) at (1,0) {$\bigtriangleup$};
        \node[text=red] (A2) at (2.1,0) {$\bigtriangleup$};
        \node[text=red] (A3) at (2.6,0) {$\bigtriangleup$};
        \node[text=red] (A4) at (3.1,0) {$\bigtriangleup$};
        \node[text=red] (A5) at (4.7,0) {$\bigtriangleup$};
        \node[text=red] (A6) at (5.5,0) {$\bigtriangleup$};
        \draw [->, thick] (-7,0) -- (7,0);
        \node[left] (X) at (8,0) {$x$};
        \node[text=blue] (B-5) at (-5.7,0) {$\square$};
        \node[text=blue] (B-4) at (-4.5,0) {$\square$};
        \node[text=blue] (B-3) at (-3.1,0) {$\square$};
        \node[text=blue] (B-2) at (-2.5,0) {$\square$};
        \node[text=blue] (B-1) at (-1.3,0) {$\square$};
        \node[text=blue] (B0) at (0.4,0) {$\square$};
        \node[text=blue] (B1) at (1.5,0) {$\square$};
        \node[text=blue] (B2) at (3.2,0) {$\square$};
        \node[text=blue] (B3) at (4.5,0) {$\square$};
        \node[text=blue] (B4) at (6.5,0) {$\square$};
        \draw[very thick] (-2.5,-1) -- (-3,-1) -- (-3,1) -- (-2.5,1);
        \draw[very thick] (2.5,-1) -- (3,-1) -- (3,1) -- (2.5,1);
        \node (J) at (0,1) {$J$};
        \node[text=blue] (B) at (7,1.5) {$\square$};
        \node (Bname) at (7.5,1.5) {$T_{1}$};
        \node[text=red] (A) at (7,1.1) {$\bigtriangleup$};
        \node (Aname) at (7.5,1.1) {$T_{2}$};
        \end{tikzpicture}
        \caption{${|J|}_{T_1} < {|J|}_{T_2}$を満たす場合のイメージ図:横軸は数直線,青色の\textcolor{blue}{$\square$}は数値型($T_{1}$)の表せる数,赤色の\textcolor{red}{$\bigtriangleup$}は数値型($T_{2}$)の表せる数}
\end{figure}
式(\ref{eq:density_compare})を満たすような範囲$J$において数値計算を行う場合,密度の高い数値型$T_{2}$の方が数値型$T_{1}$よりも計算の際に生じる誤差が少ないと予想される.
つまり,$J$において数値型$T_1$で行なった計算と,数値型$T_2$で行なった計算の誤差をそれぞれ$e_1$,$e_2$とすると,
\begin{equation}
    e_2 < e_1
\end{equation}
となると考えられる.
次章で行う数値実験は,倍精度浮動小数点(Float64)よりも64bitの固定小数点の方が密度が高いと思われる範囲での数値計算を行った.
