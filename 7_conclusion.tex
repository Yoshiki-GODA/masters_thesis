\chapter{結論}
\label{chap:結論}
\section{まとめ}
本論文では,64bitの固定小数点による計算負荷が少なく同じく64bitの倍精度浮動小数点(Float64)よりも精度の良い数値計算を提案した.
固定小数点を用いて精度の高い計算を行えるように用いる固定小数点の小数部のbit数をできる限り大きくすることにより,丸め誤差の影響が小さくなるように工夫を行なった.
結果として,第\ref{chap:数値実験}章の図\ref{fig:rossler_2^(-7)_euler_error}から図\ref{fig:rossler_2^(-13)_rk4_exact_error}までのRossler方程式や図\ref{fig:brusselator_2^(-7)_euler_error}から図\ref{fig:brusselator_2^(-13)_rk4_exact_error}までのBrusselator方程式の数値実験では,陽的Euler法,4段4次陽的Runge--Kutta法において固定小数点を用いた演算で倍精度浮動小数点(Float64)よりも精度の高い計算を行えることを示すことができた.
しかし,図\ref{fig:lorenz_2^(-7)_euler_error}から図\ref{fig:lorenz_2^(-13)_euler_exact_error}までのLorenz方程式において固定小数点を用いた場合,倍精度浮動小数点(Flaot64)と同じ程度の制度となり,精度の向上はあまり見られなかった.
だが,陽的euler法,4段4次陽的Runge--Kutta法という常微分方程式の代表的な数値解法に対して,固定小数点演算の方が有利であると考えられる範囲においては,浮動小数点演算と同程度またはそれ以上の精度で計算を行えることを示し,固定小数点の実用性を示せたと考える.
一方,固定小数点の小数部のbit数を何bitとするかについては,数値実験を繰り返し行い決めていたため,倍精度浮動小数点よりも精度の高い計算結果を得るための試行回数が多くなってしまった.

\section{今後の課題}
本論文の実験では,固定小数点を用いた数値計算の誤差に焦点を当てて実験を行った.
固定小数点による数値計算の利点としては浮動小数点に比べ計算負荷が少ないことを述べた.
そのため,理想としては固定小数点での計算が浮動小数点での計算よりも計算負荷が低いことを示したい.
しかし,実験で用いた計算機およびプロブラミング言語Juliaは固定小数点による演算を行えるようになってはいるものの,固定小数点による演算のために最適化された計算機,プログラミング言語ではないため,計算負荷を比較することができなかった.
固定小数点のために最適化された計算機やプログラミング言語を用いた実験での計算負荷の測定の検証を行う行うことは課題であると考える.


また,数値実験では固定小数点で精度の高い計算結果を得るために小数部のbit数を何bitにするかを複数回実験することにより求めた.
計算を行う前に小数部のbit数を決めることができるような理論やアルゴリズムの構築ができれば固定小数点による演算の効率をさらに高めることができると考える.