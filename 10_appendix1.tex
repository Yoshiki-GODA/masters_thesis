\chapter{補遺}
\label{chap:補遺}
\section{微分方程式の数値解法}
独立変数を$t \in \R$,未知関数$\bm{x}(t) = \left(x_1(t),x_2(t),\cdots,x_m(t)\right) \in \R^m$を$k$階微分可能であるとする.
そして,$\R^{m+1}$から$\R^m$への関数$\bm{f}(t,\bm{x}(t))$
\begin{align}
    \bm{f}(t,\bm{x}(t)) = \left(f_1(t,\bm{x}(t)),f_2(t,\bm{x}(t)),\cdots,f_m(t,\bm{x}(t))\right)
\end{align}
が与えられているとする.
また,$\bm{x}(0)$が与えられているとする.
このとき,
\begin{align}
    \differencial{\bm{x}(t)}{t} = \left(x_1^{\prime}(t),x_2^{\prime}(t),\cdots,x_m^{\prime}(t)\right)
\end{align}
として,
\begin{align}
    \differencial{\bm{x}(t)}{t} &= \bm{f}(t,\bm{x}(t)), \label{eq:ode} \\
    \bm{x}(0) &= \left(x_1(0),x_2(0),\cdots,x_m(0)\right)
\end{align}
を解くことを考える.このような問題を微分方程式の初期値問題という.
計算機を用いてこの問題を解く方法として,以下のようなものが挙げられる.
\subsection{陽的Euler法}
陽的Euler法について説明する.
$|\Delta t| << 1 $となるような十分に小さい値の$\Delta t$をとる.
このとき,
\begin{align}
    \label{eq:dtneq}
    \differencial{\bm{x}(t)}{t} \backsimeq \frac{\bm{x}(t+\Delta t) - \bm{x}(t)}{\Delta t}
\end{align}
と近似することができる.
式(\ref{eq:ode}),(\ref{eq:dtneq})より,式変形を行い
\begin{align}
    \frac{\bm{x}(t + \Delta t) - \bm{x}(t)}{\Delta t} \backsimeq \bm{f}(t,\bm{x}(t)),\\
    \bm{x}(t + \Delta t) \backsimeq \bm{x}(t) + \Delta t \bm{f}(t,\bm{x}(t)) \label{eq:euler1}
\end{align}
を得る.
この式により,時刻$t$のときの$\bm{x}(t),\bm{f}(t,\bm{x}(t))$の値から時刻$t+\Delta t$のときの$\bm{x}(t+\Delta t)$の値を計算することが可能となる.
式(\ref{eq:euler1})の左辺の$t$を$\Delta t$に置き換えることにより時刻$t + 2\delt$のときの$\bm{x}(t+2\delt),\bm{f}(t+2\delt,\bm{x}(t+2\delt))$の値を時刻$t+\Delta t$のときの$\bm{x}(t+\Delta t),\bm{f}(t+\Delta t,\bm{x}(t+\Delta t))$の値から計算することが可能となる.
この工程を繰り返すことにより,$t_k = t + k\delt \ (k = 1,2,\cdots)$とおくと,
\begin{align}
    \bm{x}(t_k) \backsimeq \bm{x}(t{k-1}) + \delt\bm{f}(t_{k-1},\bm{x}(t_{k-1})) \ \left(k = 1,2,\cdots\right) \label{eq:euler2}
\end{align}
となり時刻$t_k$の$\bm{x}(t_k)$値を計算できる.
この式(\ref{eq:euler2})を用いて微分方程式(\ref{eq:ode})の数値計算を行う方法を陽的Euler法という.

\subsection{陽的Runge--Kutta法}
陽的Runge--Kutta法について説明する.
陽的Runge--Kutta法も,陽的Euler法と同じように,$|\Delta t| << 1 $となるような十分に小さい値の$\Delta t$に対して,$\bm{x}(t),\bm{f}(t,\bm{x}(t))$から,$\bm{x}(t + \delt )$の値を計算する方法である.
以下の手順により,$\bm{x(t + \delt)}$を求める.
\begin{align}
    r_1 &= \bm{f}(\bm{x}(t),t), \\
    r_2 &= \bm{f}(\bm{x}(t)+r_1 \frac{\delt}{2},t + \frac{\delt}{2}), \\
    r_3 &= \bm{f}(\bm{x}(t)r_2 \frac{\delt}{2}, t + \frac{\delt}{2}), \\
    r_4 &= \bm{f}(\bm{x}(t)r_3 s, t + \delt), \\
    \bm{x}(t + \delt) &= \bm{x}(t) + \frac{\delt}{6}(r_1 + 2r_2 + 2r_3 + r_4).
\end{align}
上記の式により,$t_k = t + k\delt \ (k = 1,2,\cdots)$に対して,時刻$t_k$の$\bm{x}(t_k)$の値を計算できる.


\subsection{Lotka--Volterraの保存量を保存する数値解法}
Lotka--Volterraの保存量を保存する数値解法について説明する.
Lotka--Volterraは
\begin{align}
    \label{eq:lv1}
    \differencial{x}{t} &= a x - b xy,\\
    \label{eq:lv2}
    \differencial{y}{t} &= c xy - d y
\end{align}
で表される微分方程式であり,保存量は
\begin{equation}
    \label{eq:H}
    H(x,y) = cx(t) + by(t) - d\log\left(x(t)\right) - a\log\left(y(t)\right)
\end{equation}
である.
式(\ref{eq:H})を$t$で微分すると,
\begin{align}
    \differencial{H}{t} &= \differencial{}{t}\left(cx(t) + by(t) - d\log\left(x(t)\right) - a\log\left(y(t)\right)\right) \\
    &= c\differencial{x}{t} + b\differencial{y}{t} - d\differencial{\log\left(x(t)\right)}{t} - a\differencial{\log\left(y(t)\right)}{t} \\
    &= c\differencial{x}{t} + b\differencial{y}{t} - \frac{d}{x(t)}\differencial{x}{t} - \frac{a}{y(t)}\differencial{y}{t} \\
    &= c\left(ax - bxy\right) + b\left(cxy - dy\right) - \frac{d}{x}\left(ax - bxy\right) - \frac{a}{y}\left(cxy - dy\right) \\
    &= acx - bcxy + bcxy - bdy - ad + bdy - acx + ad \\
    &= 0
\end{align}
となり,$H(x,y)$は時刻$t$に依らず常に定数になる.
また,
\begin{align}
    \nabla H = \begin{bmatrix}
        \partialdiff{H}{x} \\  \\
        \partialdiff{H}{y}
    \end{bmatrix} =
    \begin{bmatrix}
        c - \frac{d}{x} \\
        b - \frac{a}{y}
    \end{bmatrix}
\end{align}
となり,式(\ref{eq:lv1}),(\ref{eq:lv2})より,
\begin{align}
    \label{eq:lv3}
    \begin{bmatrix}
        \differencial{x}{t} \\ 
        \differencial{y}{t}
    \end{bmatrix}
    = 
    \begin{bmatrix}
        0 & -xy \\
        xy & 0
    \end{bmatrix}
    \nabla H
\end{align}
となる. 
この式から,時刻$t_i$における$x,y$の値$x_i,y_i$から$\delt$だけ更新した時刻$t_{i+1} = (t + \delt)$における$x,y$の値$x_{i+1},y_{i+1}$を計算する.
式(\ref{eq:lv3})の左辺を,
\begin{align}
    \begin{bmatrix}
        \differencial{x}{t} \\ 
        \differencial{y}{t}
    \end{bmatrix}
     \backsimeq 
    \begin{bmatrix}
        \frac{x_{i+1} - x_i}{\delt} \\
         \frac{y_{i+1} - y_i}{\delt}
    \end{bmatrix}
\end{align}
と近似し,$\nabla H$の近似$\bar{\nabla}H$を
\begin{align}
    \bar{\nabla}H = 
    \begin{bmatrix}
        \frac{H(x_{i+1},y_{i+1}) - H(x_i,y_{i+1})}{x_{i+1} - x_i} \\
        \frac{H(x_{i},y_{i+1}) - H(x_i,y_i)}{y_{i+1} - y_i}
    \end{bmatrix}
\end{align}
とする.
式(\ref{eq:lv3})より,
\begin{align}
    \begin{bmatrix}
        \frac{x_{i+1} - x_i}{\delt} \\
         \frac{y_{i+1} - y_i}{\delt}
    \end{bmatrix}
    = 
    \begin{bmatrix}
        0 & -x_i y_i \\
        x_i y_i & 0
    \end{bmatrix}
    \begin{bmatrix}
        \frac{H(x_{i+1},y_{i+1}) - H(x_i,y_{i+1})}{x_{i+1} - x_i} \\
        \frac{H(x_{i},y_{i+1}) - H(x_i,y_i)}{y_{i+1} - y_i}
    \end{bmatrix}
\end{align}
という式を得る.
この式を$x_{i+1},y_{i+1}$について解く.