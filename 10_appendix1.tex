\chapter{補遺}
\label{chap:補遺}
\section{微分方程式の数値解法}
独立変数を$t \in \R$,未知関数$\bm{x}(t) = \left(x_1(t),x_2(t),\cdots,x_m(t)\right) \in \R^m$を$k$階微分可能であるとする.
そして,$\R^{m+1}$から$\R^m$への関数$\bm{f}(t,\bm{x}(t))$
\begin{align}
    \bm{f}(t,\bm{x}(t)) = \left(f_1(t,\bm{x}(t)),f_2(t,\bm{x}(t)),\cdots,f_m(t,\bm{x}(t))\right)
\end{align}
が与えられているとする.
また,$\bm{x}(0)$が与えられているとする.
このとき,
\begin{align}
    \frac{d \bm{x}(t)}{dt} = \left(x_1^{\prime}(t),x_2^{\prime}(t),\cdots,x_m^{\prime}(t)\right)
\end{align}
として,
\begin{align}
    \frac{d \bm{x}(t)}{dt} &= \bm{f}(t,\bm{x}(t)), \label{eq:ode} \\
    \bm{x}(0) &= \left(x_1(0),x_2(0),\cdots,x_m(0)\right)
\end{align}
を解くことを考える.このような問題を微分方程式の初期値問題という.
計算機を用いてこの問題を解く方法として,以下のようなものが挙げられる.
\subsection{陽的Euler法}
陽的Euler法について説明する.
$|\Delta t| << 1 $となるような十分に小さい値の$\Delta t$をとる.
このとき,
\begin{align}
    \frac{d \bm{x}(t)}{dt} \backsimeq \frac{\bm{x}(t+\Delta t) - \bm{x}(t)}{\Delta t}
\end{align}