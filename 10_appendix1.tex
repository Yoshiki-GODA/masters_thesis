\chapter{補遺}
\label{chap:補遺}
\section{微分方程式の数値解法}
独立変数を$t \in \R$,未知関数$\bm{x}(t) = \left(x_1(t),x_2(t),\cdots,x_m(t)\right) \in \R^m$を$k$階微分可能であるとする.
そして,$\R^{m+1}$から$\R^m$への関数$\bm{f}(t,\bm{x}(t))$
\begin{align}
    \bm{f}(t,\bm{x}(t)) = \left(f_1(t,\bm{x}(t)),f_2(t,\bm{x}(t)),\cdots,f_m(t,\bm{x}(t))\right)
\end{align}
が与えられているとする.
また,$\bm{x}(0)$が与えられているとする.
このとき,
\begin{align}
    \frac{d \bm{x}(t)}{dt} = \left(x_1^{\prime}(t),x_2^{\prime}(t),\cdots,x_m^{\prime}(t)\right)
\end{align}
として,
\begin{align}
    \frac{d \bm{x}(t)}{dt} &= \bm{f}(t,\bm{x}(t)), \label{eq:ode} \\
    \bm{x}(0) &= \left(x_1(0),x_2(0),\cdots,x_m(0)\right)
\end{align}
を解くことを考える.このような問題を微分方程式の初期値問題という.
計算機を用いてこの問題を解く方法として,以下のようなものが挙げられる.
\subsection{陽的Euler法}
陽的Euler法について説明する.
$|\Delta t| << 1 $となるような十分に小さい値の$\Delta t$をとる.
このとき,
\begin{align}
    \label{eq:dtneq}
    \frac{d \bm{x}(t)}{dt} \backsimeq \frac{\bm{x}(t+\Delta t) - \bm{x}(t)}{\Delta t}
\end{align}
と近似することができる.
式(\ref{eq:ode}),(\ref{eq:dtneq})より,式変形を行い
\begin{align}
    \frac{\bm{x}(t + \Delta t) - \bm{x}(t)}{\Delta t} \backsimeq \bm{f}(t,\bm{x}(t)),\\
    \bm{x}(\Delta t) \backsimeq \bm{x}(t) + \Delta t \bm{f}(t,\bm{x}(t)) \label{eq:euler1}
\end{align}
を得る.
この式により,時刻$t$のときの$\bm{x}(t),\bm{f}(t,\bm{x}(t))$の値から時刻$t+\Delta t$のときの$\bm{x}(t+\Delta t),\bm{f}(t+\Delta t,\bm{x}(t+\Delta t))$の値を計算することが可能となる.
式(\ref{eq:euler1})の左辺の$t$を$\Delta t$に置き換えることにより時刻$t + 2\delt$のときの$\bm{x}(t+2\delt),\bm{f}(t+2\delt,\bm{x}(t+2\delt))$の値を時刻$t+\Delta t$のときの$\bm{x}(t+\Delta t),\bm{f}(t+\Delta t,\bm{x}(t+\Delta t))$の値から計算することが可能となる.
この工程を繰り返すことにより,$t_k = t + k\delt$とおくと,
\begin{align}
    \bm{x}(t_k) \backsimeq \bm{x}(t{k-1}) + \delt\bm{f}(t_{k-1},\bm{x}(t_{k-1})) \ \left(k = 1,2,\cdots\right) \label{eq:euler2}
\end{align}
となり時刻$t_k$の$\bm{x}(t_k)$値を計算できる.
この式(\ref{eq:euler2})を用いて微分方程式(\ref{eq:ode})の数値計算を行う方法を陽的Euler法という.

\section{陽的Runge-Kutta法}
陽的Runge-Kutta法について説明する.

\section{St\"ormer-Verlet法}
St\"ormer-Verlet法について説明する.
St\"ormer-Verlet法は,保存量を保存するように数値解を計算する方法である.
