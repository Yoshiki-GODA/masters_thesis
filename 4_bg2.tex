\chapter{丸め誤差・情報落ち・桁落ちについての基礎知識}
前章で述べたように計算機の中では表現できる数に限りがある.
そのため計算機の中で表現できる数はそのままの数で計算を行うが,計算機の中で表現できない数はその数にできるだけ近い表現をできる数を代わりに用いて計算を行うことになる.
ある数$a\ \in \R$に対して,計算機で$a$という数を扱う場合の表現を
\begin{equation*}    
    a^{\ast} \ (\in \mathbb{F})
\end{equation*}
とする.
また,$a$を$p$進数で表現することを明記する場合,
\begin{equation*}
    a_{(p)}
\end{equation*}
と表すことにする.
\label{chap:基礎知識2}
\section{丸め誤差}
丸め誤差について説明する.
丸め誤差とは,真の値と計算機で表現するために代わりに用いる数の差である.
真の値を$x_{\mathrm{true}} \ (\in \R)$とする.
計算機の中で表現できる数で$x_{\mathrm{true}}$に一番近い数を$x^{\ast} \ (\in \mathbb{F})$とすると,そのときの丸め誤差$e_{round}$は,
\begin{equation*}
    e_{\mathrm{round}} = x_{\mathrm{true}} - x^{\ast}
\end{equation*}
と表せる.
簡単のため,計算機で扱われる2進数ではなく10進数を用いた場合を考える.
例えば,計算機では,小数点2桁までしか表現できないとすると,$\pi$という数は,計算機の中で
\begin{equation*}
    \pi^{\ast}_{(10)} = 3.14_{(10)} \ (\in \mathbb{F})
\end{equation*}
という数で扱うことになる.
そして,丸め誤差$e_{\mathrm{round}}$は,
\begin{equation*}
    {e_{\mathrm{round}}}_{(10)} = \pi - \pi^{\ast} = {0.00159265_{(10)}\dots}
\end{equation*}
となる.


次に,丸め誤差の大きさがどの程度であるかについて述べる.
$a \in \R$を計算機の中で表現することを考える.ただし,ここでは計算機の中で表現される数は10進数ではなく2進数であるとする.
ただし,浮動小数点で表せる最大の数を${M_{\mathrm{float}}}_{(2)}$,最小の数を${m_{\mathrm{float}}}_{(2)}$,固定小数点で表せる最大の数を${M_{\mathrm{fixed}}}_{(2)}$,最小の数を${m_{\mathrm{fixed}}}_{(2)}$とし,
\begin{align*}
   {m_{\mathrm{float}}}_{(2)} &< a_{(2)} < {M_{\mathrm{float}}}_{(2)}, \\
   {m_{\mathrm{fixed}}}_{(2)} &< a_{(2)} < {M_{\mathrm{fixed}}}_{(2)}
\end{align*}
とする.
浮動小数点での丸め誤差について述べ,続いて固定小数点の丸め誤差について述べる.
一般的な浮動小数点(Float32,Float54)では,$m_{\mathrm{float}}$,$M_{\mathrm{float}}$のおおよその値は以下のような値である.
\begin{table}[H]
    \centering
    \caption{Float32,Float64での$m_{\mathrm{float}}$と$M_{\mathrm{float}}$のおおよその値.}
    \begin{tabular}{c|c|c}
         & $m_{\mathrm{float}}$ & $M_{\mathrm{float}}$ \\ \hline \hline
         Float32 & $-3.40 \times 10^{38}$ & $3.40 \times 10^{38}$ \\ \hline
         Float64 & $-1.80 \times 10^{308}$ & $1.80 \times 10^{308}$
    \end{tabular}
\end{table}

丸め誤差は浮動小数点の場合,仮数部を$m$bit,指数部を$e$bitとする.
ただし,以下ではFloat32,Float64とは異なり指数部$e$にバイアスはないものとする.
$a^{\ast}_{(2)} \ (\in \flonum)$の中で,
\begin{align*}
    {a^{\ast}}_{(2)} < a_{(2)}\ \text{かつ} |a_{(2)} - {a^{\ast}}_{(2)}| \ \text{が最小}
\end{align*}
となるような数${a^{\ast}}_{(2)}$を${a^{\ast}_{\mathrm{small}}}_{(2)}$とする.
${a^{\ast}_{\mathrm{small}}}_{(2)}$の仮数部を$\mathcal{M}_{\mathrm{small}}$,指数部を$\mathcal{E}_{\mathrm{small}}$とする.
たま,
\begin{align*}
    {a^{\ast}}_{(2)} > a_{(2)} \ \text{かつ} |a_{(2)} - {a^{\ast}}_{(2)}| \ \text{が最小}
\end{align*}
となるような数${a^{\ast}}_{(2)}$を${a^{\ast}_{\mathrm{large}}}_{(2)}$とする.
${a^{\ast}_{\mathrm{large}}}_{(2)}$の仮数部を$\mathcal{M}_{\mathrm{large}}$,指数部を$\mathcal{E}_{\mathrm{large}}$とする.
このとき,
\begin{align*}
    |a - {a^{\ast}_{\mathrm{sall}}}_{(2)}| &\leq 2^{-m+{\mathcal{E}_{small}}}, \\
    |a - {a^{\ast}_{\mathrm{large}}}_{(2)}| &\leq 2^{-m+{\mathcal{E}_{large}}}
\end{align*}
であり,$a^{\ast}_{(2)}$が
\begin{align*}
    2^{\beta -1} < a^{\ast}_{(2)} < 2^{\beta}
\end{align*}
となるとき,
\begin{equation*}
    \mathcal{E}_{\mathrm{small}} = \mathcal{E}_{\mathrm{large}} = \beta
\end{equation*}
である.$a^{\ast}_{(2)}$は
\begin{align*}
    {a^{\ast}}_{(2)} = \left\{ 
        \begin{array}{ll}
            {a^{\ast}_{\mathrm{small}}}_{(2)} &\text{if} \ |a^{\ast}_{2} - {a^{\ast}_{\mathrm{small}}}_{(2)} | < |a^{\ast}_{2} - {a^{\ast}_{\mathrm{large}}}_{(2)} | \\
            {a^{\ast}_{\mathrm{large}}}_{(2)} &\text{if} \ |a^{\ast}_{2} - {a^{\ast}_{\mathrm{small}}}_{(2)} | \geq |a^{\ast}_{2} - {a^{\ast}_{\mathrm{large}}}_{(2)} | 
        \end{array}
    \right.
\end{align*}
となるので,丸め誤差$e_{\mathrm{round}}$は最大で
\begin{align}
    \label{eq:rounderror_float}
    |e_{\mathrm{round}}| \leq 2^{-m+\beta -1}
\end{align}
となる.
ここで,$m$は一定の値であり,$\beta$は$a^{\ast}_{(2)}$の値により決まる値である.


固定小数点の場合,小数部が$m$bitとし,${a^{\ast}}_{(2)} \ (\in\fixnum)$の中で,
\begin{align*}
    {a^{\ast}}_{(2)} < a_{(2)}\ \text{かつ} |a_{(2)} - {a^{\ast}}_{(2)}| \ \text{が最小}
\end{align*}
となるような数${a^{\ast}}_{(2)}$を${a^{\ast}_{\mathrm{small}}}_{(2)}$とする.
たま,
\begin{align*}
    {a^{\ast}}_{(2)} > a_{(2)} \ \text{かつ} |a_{(2)} - {a^{\ast}}_{(2)}| \ \text{が最小}
\end{align*}
となるような数${a^{\ast}}_{(2)}$を${a^{\ast}_{\mathrm{large}}}_{(2)}$とする.
このとき,
\begin{align}
    |a - {a^{\ast}_{\mathrm{sall}}}_{(2)}| &\leq 2^{-m}, \\
    |a - {a^{\ast}_{\mathrm{large}}}_{(2)}| &\leq 2^{-m}
\end{align}
であり,
\begin{align*}
    {a^{\ast}}_{(2)} = \left\{ 
        \begin{array}{ll}
            {a^{\ast}_{\mathrm{small}}}_{(2)} &\text{if} \ |a^{\ast}_{2} - {a^{\ast}_{\mathrm{small}}}_{(2)} | < |a^{\ast}_{2} - {a^{\ast}_{\mathrm{large}}}_{(2)} | \\
            {a^{\ast}_{\mathrm{large}}}_{(2)} &\text{if} \ |a^{\ast}_{2} - {a^{\ast}_{\mathrm{small}}}_{(2)} | \geq |a^{\ast}_{2} - {a^{\ast}_{\mathrm{large}}}_{(2)} | 
        \end{array}
    \right.
\end{align*}
となるので,丸め誤差$e_{\mathrm{round}}$は最大で
\begin{align}
    \label{eq:rounderror_fixed}
    |e_{\mathrm{round}}| \leq 2^{-m-1}
\end{align}
となる.ここで$m$は一定の値である.


式(\ref{eq:rounderror_float}),(\ref{eq:rounderror_fixed})から,浮動小数点と固定小数点では丸め誤差の大きさが異なる.
特に,固定小数点では丸め誤差取りうる値のうち最大値が$a^{\ast}_{(2)}$の値に依らず一定であるのに対して,浮動小数点では$a^{\ast}_{(2)}$の値がどのような範囲にあるのかにより丸め誤差がとりうる値のうち最大値の大きさが異なる.


計算機で計算を繰り返し行うと,一度の計算で発生した丸め誤差が次の計算に影響を与えるため,計算回数が多くなるにつれ計算機で行われる計算の結果と厳密な計算の結果の差は大きくなる.

\section{打ち切り誤差}
打ち切り誤差とは,問題の真の結果と解法を用いて厳密な計算により得られる結果の差である.
以下のような微分方程式を陽的Euler法で$t \in [0,T]$の範囲で解くとこを考える:
\begin{equation*}
    \differencial{x}{t} = f(t,x(t)), \ x(0) = x_0.
\end{equation*}
Euler法は付録\ref{chap:補遺}で説明する方法である.
まず,時刻$t,t+\Delta t$における解を$x(t),x(t+\Delta t)$とし,2次の項までTaylor展開すると
\begin{align}
    \label{eq:taylor_euler}
    x(t + \Delta t) = x(t) + \frac{\delt}{1!}x^{\prime}(t) + \frac{\delt^2}{2!}x^{\prime\prime}(t + \theta\delt) \ (0 < \theta < 1)
\end{align}
である.
一方,陽的Euler法では
\begin{align}
    \label{eq:euler_appro}
    x(t + \Delta t) \backsimeq x(t) + \frac{\delt}{1!}f(t,x(t))
\end{align}
という近似により解を得るため,式(\ref{eq:taylor_euler}),(\ref{eq:euler_appro}より真の解$x_{\mathrm{true}}(t)$とEuler法により得られる解$x_{\mathrm{euler}}(t)$には,
\begin{align*}
    x_{\mathrm{true}}(t) - x_{\mathrm{euler}}(t) = \frac{\delt^2}{2!}x^{\prime\prime}(t + \theta\delt) \ (0 < \theta < 1)
\end{align*}
という誤差が含まれる.
上の例では,任意の$t$に対して$t$の十分近くでは$|x^{\prime\prime}(t)| \leq M$となるような$M \in \R$が存在していると仮定すると,更新幅$\delt$を小さくすることにより打ち切り誤差を小さくできる.
しかし,$\delt$を小さくすると終端時刻$T$までに計算を行う回数が多くなる.
加えて,各計算ごとに丸め誤差が生じると考えると,更新した値を求めるたびに蓄積する誤差の影響が大きくなる.
そのため,打ち切り誤差と丸め誤差はトレードオフの関係にある.
\begin{figure}[H]
    \centering
    \begin{tikzpicture}
        \node [below left] at (0,0) {$O$};
        \draw [->, thick] (-1,0) -- (7,0);
        \node [right] at (7,0) {更新幅$\delt$};
        \draw [->, thick] (0,-1) -- (0,4);
        \node [left] at (0,4) {全体の誤差};
        \draw [thick, blue] (0,3.5) -- (3.5,1) -- (7,3.5);
        \draw [thick, dashed, blue] (0,3.3) -- ++(4.2,-3);
        \draw [thick,dotted] (0.1,3.2) to [out=290,in=180] (3.4,0.9);
        \node [fill=white, below, font=\scriptsize] at (1.15,1.7) {丸め誤差の蓄積};
        \draw [thick, dashed, blue] (7,3.3) -- ++(-4.2,-3);
        \draw [thick, dotted] (3.6,0.9) to [out=10,in=250] (6.9,3.2);
        \node [fill=white, below, font=\scriptsize] at (5.85,1.7) {打ち切り誤差};
    \end{tikzpicture}
    \caption{同じ終端時刻$T$まで計算を行うときの更新幅$\delt$と打ち切り誤差と丸め誤差の関係.}
    \label{fig:error_tradingoff}
\end{figure}

\section{情報落ち}
情報落ちとは,浮動小数点数を用いて非常に離れた値の数同士を計算する際に生じる現象である.
簡単のため,計算機で扱われる2進数ではなく10進数を用いた場合を考える.
浮動小数点数の仮数部が小数点3桁までしか表現できない場合を考える.
a,bを浮動小数点数で表せる数として,$|a| >> |b|$とする.例えば
\begin{align*}
    a &= 1.000 \times 10^{-2}, \\
    b &= 1.000 \times 10^{3}
\end{align*}
として,
\begin{equation*}
    a + b
\end{equation*}
を考えると,
\begin{equation*}
    a + b = 1.001 \times 10^{-2} + 1.000 \times 10^{3} = 0.01 + 1000 = 1000.01 \backsimeq 1.000 \times 10^{3}
\end{equation*}
となり,$a$を足したという情報が失われてしまうことになる.
このような現象を情報落ちという.

\section{桁落ち}
桁落ちとは,浮動小数点数を用いて非常に近い数同士を計算する際に生じる現象である.
簡単のため,計算機で扱われる2進数はなく10進数を用いた場合を考える.
浮動小数点数の仮数部が小数点3桁までしか表現できない場合を考える.
a,bを浮動小数点数で表せる数として,$|a - b| << 1$とする.
例えば,
\begin{align*}
    a &= 1.234 \times 10^{-2}, \\
    b &= 1.233 \times 10^{-2}
\end{align*}
として,
\begin{equation*}
    a - b
\end{equation*}
を考えると,
\begin{align*}
    a - b &= 1.234 \times 10^{-2} - 1.233 \times 10^{-2} = 0.01234 - 0.01233 \\
    &= 0.00001 = 1.000 \times 10^{-5}
\end{align*}
となり,計算前の$a,b$は仮数部に4桁の情報があったのに対し,計算後の$a-b$の結果は仮数部の情報が1桁になってしまっている.
このような現象を桁落ちという.