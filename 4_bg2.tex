\chapter{丸め誤差・桁落ちについての基礎知識}
\label{chap:基礎知識2}
\section{丸め誤差}
丸め誤差について説明する.
丸め誤差とは,計算機を用いて数を計算する際に用いる誤差である.
前章で説明したように計算機で用いられる数は,表現出来る数に限りがある.
そのため,計算機の中で表現できない数を計算の過程で用いないといけない場合,計算機ではその数に近い値を用いて表す.
真の値が$x_true$であり,計算機の中で表現できる数で$x_{true}$い一番近い数を$x$とすると,そのときの丸め誤差$e_{round}$は,
\begin{equation}
    e_{round} = x_{true} - x
\end{equation}
と表せる.
例えば,計算機では,小数点2桁までしか表現できないとすると,$\pi$という数は,計算機の中で
\begin{equation}
    \pi^{\ast} = 3.14
\end{equation}
という数で扱うことになる.
そして,丸め誤差$e_{round}$は,
\begin{equation}
    e_{round} = \pi - \pi^{\ast} = 0.00159265\dots
\end{equation}
となる.
丸め誤差は,浮動小数点の場合 \\
固定小数点の場合

\section{桁落ち}