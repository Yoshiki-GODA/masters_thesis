\chapter{序論}
\label{chap:序論}
研究背景,研究目的,本論文の構成について述べる.
\begin{comment}
    目標規定:
    この修士論文は,固定小数点演算が浮動小数点演算に比べ精度の高い計算を行える場合があることを示すために書く.
\end{comment}
\section{研究背景}
現在の計算機の多くは2進数の浮動小数点を用いて計算を行っている.
固定小数点での計算では,浮動小数点に比べ精度が落ちるが\cite{IJERTV12IS010134}においてc言語を用いて浮動小数点よりも速く計算が行えるとこが示されている.
このようなことから,限定的ではあるが固定小数点を用いた計算が有効である可能性がある.

\section{研究目的}
本研究の目的は,固定小数点演算の有用性を示すことである.
固定小数点を使う利点としては,第\ref{chap:基礎知識1}で述べる,情報落ちや桁落ちがない点が挙げられる.
一方で,浮動小数点と比較して表現できる数の範囲が少ない,$0$に近い値では相対的に丸め誤差が大きくなるという欠点も存在する.
本研究では固定小数点の利点を活かし,欠点の影響が少ない計算を行い浮動小数点での計算と比較した.
xx\\
cc \\
cc\\
v\\
$\tilde{e} $\\
$\bar{e}$\\
$e^{\ast}$\\
$e^{\star}$\\
\section{本論文の構成}
本論文の構成は,?つの章によって構成される.
まず,第\ref{chap:序論}章では本研究の研究背景及び研究目的を述べた.次の第\ref{chap:基礎知識1}章,第\ref{chap:基礎知識2}章では固定小数点及び浮動小数点についての基礎知識を紹介する.
次に, 第\ref{chap:提案手法}章で固定小数点演算を行う利点について述べる.
第\ref{chap:数値実験}では,第\ref{chap:提案手法}で述べた固定小数点の利点を以下した数値計算を実際に行い,浮動小数点での演算の結果と比較する.
最後に,第\ref{chap:結論}章で本論文のまとめを述べる.
