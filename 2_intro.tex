\chapter{序論}
\label{chap:序論}
研究背景,研究目的,本論文の構成について述べる.
\begin{comment}
    目標規定:
    この修士論文は,固定小数点演算が浮動小数点演算に比べ精度の高い計算を行える場合があることを示すために書く.
\end{comment}
\section{研究背景}
\begin{comment}
    研究テーマにおける既存の知見は何か?
    分かっていないこと、究明すべきことは何か?
    その不足点を究明する意義は何か?
    自分の研究の論拠や仮説は何か?
    
    ストーリー:
    固定小数点演算は浮動小数点演算よりも計算時間が速いとされている.
    しかし,浮動小数点に比べて精度が低くなる恐れがある.
    本論文では,固定小数点と浮動小数点での数値計算の結果を比べ,固定小数点での演算も浮動小数点と同程度の精度で計算できることを示した.
    実験結果より固定小数点演算で浮動小数点より高速で,浮動小数点と同程度の計算を実現できるのではないかと考える.
\end{comment}
現在の計算機の多くは2進数の浮動小数点を用いて計算を行っている.
計算機の中で用いられている浮動小数点IEEE((米国)電子電気学会)が1985年に定めたIEEE754という規格に基づいている.
1980年代以前は,計算機ごとに独自の規格の浮動小数点や固定小数点が用いれれていたが,IEEE754という標準規格が定められたことにより,多くのプログラミング言語ではこの基準に基づいた浮動小数点での演算を行うように設計されている.
現在は多くの計算機において浮動小数点が主流であるが,一般に固定小数点は浮動小数点よりも計算を高速で行えるため,浮動小数点が提案されて以降も車載コンピュータなどに用いらていた.
また,近年の研究でc言語を用いて浮動小数点よりも速く計算が行えるとこが示されている\cite{IJERTV12IS010134}.
また,\cite{pmlr-v48-linb16}では機械学習を固定小数点演算を用いて実装することにより精度を損なわずにモデルサイズを減らすことができることを示している.
このように,固定小数点を浮動小数点で行われている計算に用いることにより,計算の速度を向上する可能性がある.
しかし,固定書数点は同じbit数を持つ浮動小数点に比べ表現できる数の範囲が狭いため,計算を行う数の範囲に気をつける必要がある.
加えて浮動小数点の仮数部と固定小数点の小数部のbit数が同じである場合$0$に非常に近い数を表す際に生じる誤差が浮動小数点に比べて大きくなるという問題がある.


\section{研究目的}
本研究の目的は,同じbit数の固定小数点と浮動小数点を比較し固定小数点演算の有用性を示すことである.
一般に同じbit数の固定小数点と浮動小数点では,表現できる数の範囲や誤差の大きさに差があり浮動小数点を用いて計算した方が計算精度が高いとされる.
しかし,固定小数点を用いて計算が浮動小数点を用いて行なった計算と比較して同程度または浮動小数点よりも精度の高い計算結果を得られたならば,固定小数点は高速かつ精度の高い計算を行えることになる.
本研究では,微分方程式に固定小数点を用いた計算と浮動小数点を用いた計算を行い比較することによって固定小数点を用いた演算によって高い精度の計算結果が得られることを示し,固定小数点演算が有用であることを提示する.


\section{本論文の構成}
本論文の構成は, 以下の通りである.
第\ref{chap:基礎知識1}章,第\ref{chap:基礎知識2}章では固定小数点及び浮動小数点についての基礎知識を紹介する.
次に, 第\ref{chap:提案手法}章で固定小数点演算を行う利点について述べる.
第\ref{chap:数値実験}では,第\ref{chap:提案手法}で述べた固定小数点の利点を以下した数値計算を実際に行い,浮動小数点での演算の結果と比較する.
最後に,第\ref{chap:結論}章で本論文のまとめを述べる.
