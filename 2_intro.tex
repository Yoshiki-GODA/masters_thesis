\chapter{序論}
\label{chap:序論}
研究背景,研究目的,本論文の構成について述べる.

\section{研究背景}
現在の計算機の多くは2進数の浮動小数点を用いて計算を行っている.
固定小数点での計算では,浮動小数点に比べ精度が落ちるが\cite{IJERTV12IS010134}においてc言語を用いて浮動小数点よりも速く計算が行えるとこが示されている.
このようなことから,限定的ではあるが固定小数点を用いた計算が有効である可能性がある.

\section{研究目的}
本研究の目的は,固定小数点演算の有用性を示すことである.
固定小数点を使う利点としては,〜が挙げられる.
一方で,〜という欠点も存在する.
本研究では固定小数点の利点を活かし,欠点の影響が少ない計算を行い浮動小数点での計算と比較した.
xx\\
cc \\
cc\\
v\\
$\tilde{e} $\\
$\bar{e}$\\
$e^{\ast}$\\
$e^{\star}$\\
\section{本論文の構成}
本論文の構成は,?つの章によって構成される.
まず,第\ref{chap:序論}章では本研究の研究背景及び研究目的を述べた.次の第\ref{chap:基礎知識1}章,第\ref{chap:基礎知識2}章では固定小数点及び浮動小数点についての基礎知識を紹介する.
次に\cite{hopkins2020stochastic}
