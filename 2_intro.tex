\chapter{序論}
\label{chap:序論}
研究背景,研究目的,本論文の構成について述べる.

\section{研究背景}

現在の計算機の多くは2進数の浮動小数点を用いて計算を行っている.
計算機の中で用いられている浮動小数点IEEE((米国)電子電気学会)が1985年に定めたIEEE754という規格に基づいている.
1980年代以前は,計算機ごとに独自の規格の浮動小数点や固定小数点が用いられていたが,IEEE754という標準規格が定められたことにより,多くのプログラミング言語ではこの基準に基づいた浮動小数点での演算を行うように設計されている.
現在は多くの計算機において浮動小数点が主流であるが,一般に固定小数点は浮動小数点よりも計算を高速で行えるため,浮動小数点が提案されて以降も車載コンピュータなどに用いらていた.
また,近年の研究でC言語を用いて浮動小数点よりも速く計算が行えるとこが示されている\cite{IJERTV12IS010134}.
近年では,機械学習を固定小数点演算を用いて実装することにより精度を損なわずにモデルサイズを減らすことができることを示している\cite{pmlr-v48-linb16}.
このように,固定小数点を浮動小数点で行われている計算に用いることにより,計算の速度を向上する可能性がある.
しかし,固定書数点は同じbit数を持つ浮動小数点に比べ表現できる数の範囲が狭いため,計算を行う数の範囲に気をつける必要がある.


先行研究では,小数部が11bitの固定小数点を用いて,浮動小数点よりも速い計算時間で連立方程式や高次方程式の近似解を求めるている\cite{IJERTV12IS010134}.
しかし,\cite{IJERTV12IS010134}において,精度に関しては浮動小数点の方が精度が高いことも示されている.
一方で,32bitの固定小数点を用いて微分方程式を解き32bitの単精度浮動小数点(Float32)と比較し,浮動小数点と同程度の精度の良い結果を得られるという報告もある\cite{hopkins2020stochastic}.
\cite{hopkins2020stochastic}では,計算時に計算機にかかる負荷が固定小数点の方が浮動小数点比べ低いことは示唆されているが,使用した固定小数点はISO18037の規格に従った数値型(s16.15,u0.32,s0.31,s8.7,u0.16,s0.15)であり,いずれも32bit以下の固定小数点である.
現在,多くのプログラミング言語では通常使われる浮動小数点は64bitの倍精度浮動小数点(Float64)であり,32bitの単精度浮動小数点(Float32)よりも精度が高い計算が行える.
そこで,倍精度浮動小数点(Float64)と同じ64bitの固定小数点を用い,また演算が行える範囲内で最大限小数部のbit数を増やすことによって倍精度浮動小数点(Float64)よりも計算の精度を高めることができるのではないかと考える.
固定小数点を用いた計算をサポートしている言語(C,C++等)はまだ少なく,多くの数値計算では浮動小数点を用いるのが一般的であるが,将来的に固定小数点での演算のために最適化された計算機やプログラミング言語の開発が行なわれれば,固定小数点を用いた計算負荷が少なく計算時間が短く浮動小数点よりも精度の高い計算を行えることが可能になるのではないかと考える.

\section{研究目的}
本研究の目的は,固定小数点の計算の精度に着目し,同じbit数の固定小数点と浮動小数点を比較し固定小数点演算の有用性を示すことである.
一般に固定小数点と浮動小数点では,表現できる数の範囲や誤差の大きさに差があり浮動小数点を用いて計算した方が計算精度が高いとされる.
そのため,現代の計算機では浮動小数点が使われているものがほとんどである.
しかし,固定小数点を用いて計算が浮動小数点を用いて行なった計算と比較して同程度または浮動小数点よりも精度の高い計算結果を得られたならば,固定小数点演算も実用的である.
本研究では,微分方程式に固定小数点を用いた計算と浮動小数点を用いた計算を行い比較することによって固定小数点を用いた演算によって高い精度の計算結果が得られることを示し,固定小数点演算が有用であることを提示する.

\section{本論文の構成}
本論文の構成は, 以下の通りである.
第\ref{chap:基礎知識1}章では,固定小数点及び浮動小数点の表現方法や演算方法と計算機における誤差について説明する.
次の第\ref{chap:提案手法}では,固定小数点の性能が発揮できると考えられる状況について述べ,その状況下で固定小数点の実用性を示すための方法を述べる.
第\ref{chap:数値実験}章では,第\ref{chap:提案手法}章で述べた固定小数点の性能が発揮できる状況下で数値計算を行い,浮動小数点での演算の結果と比較し,誤差を浮動小数点と同程度またはより小さくできることを示す.
最後に,第\ref{chap:結論}章で本論文のまとめを述べる.
