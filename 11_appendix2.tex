\chapter{実験環境とプログラムコード}
\label{chap:実験環境とプログラムコード}
\section{実験環境}
本論文の数値実験を行った実験環境は表\ref{tab:machine_spec}の通りである.
\begin{table}[H]
    \centering
    \caption{実験環境の構成.}
    \begin{tabular}{|c||c|}
        \hline
        CPU & Apple M2\\
        \hline
        OS & MacOS Sequoia\\
        \hline
        RAM & 16GB\\
        \hline
        言語 & Jula v1.10.1\\
        \hline
    \end{tabular}
    \label{tab:machine_spec}
\end{table}


\section{プログラムコード}
数値実験を行った際に用いたプログラムを掲載する.
プログラムに用いる,パッケージは\ref{code_package}
\begin{lstlisting}[caption=package,label=code_package]{}
using FixedPointNumbers #固定小数点を使用するためのパッケージ

using DifferentialEquations #微分方程式を数値計算をするためのパッケージ

using CairoMakie #グラフを描画するためのパッケージ
\end{lstlisting}
であり,数値計算をするためのプログラムはソースコード\ref{code_numericalscheme}である.
\begin{lstlisting}[caption =\text{numerical\_scheme.jl}, label =code_numericalscheme]{}
using FixedPointNumbers

#誤差の計算
function err(x_ex,x)
    x_err = x_ex .- x
end

function err_norm_2d(x_ex,y_ex,x,y)
    x_err = x_ex .- x
    y_err = y_ex .- y
    return sqrt.((x_err).^2 .+ (y_err).^2)
end

function err_norm_3d(x_ex,y_ex,z_ex,x,y,z)
    x_err = x_ex .- x
    y_err = y_ex .- y
    z_err = z_ex .- z
    return sqrt.((x_err).^2 .+ (y_err).^2 .+ (z_err).^2)
end

#オイラー法
function euler_1(f1,x0,stepsize,T,N) 
    s = T(stepsize)
    result1 = T[x0]
    sizehint!(result1,N+1)
    S= T(0.0)
    for i in range(1,N)
        x1 = result1[end]
        S+=s
        x_new = x1 + s*f1(x1,S)
        push!(result1,x_new)
    end
    return result1
end

function euler_2(f1,f2,x0,y0,stepsize,T,N)
    s = T(stepsize)
    result1 = T[x0]
    result2 = T[y0]
    sizehint!(result1,N+1)
    sizehint!(result2,N+1)
    S= T(0.0)
    for i in range(1,N)
        x1 = result1[end]
        y1 = result2[end]
        S+=s
        x_new = x1 + s*f1(x1,y1,S)
        y_new = y1 + s*f2(x1,y1,S)
        push!(result1,x_new)
        push!(result2,y_new)
    end
    return result1, result2
end

function euler_3(f1,f2,f3,x0,y0,z0,stepsize,T,N)
    s = T(stepsize)
    result1 = T[x0]
    result2 = T[y0]
    result3 = T[z0]
    sizehint!(result1,N+1)
    sizehint!(result2,N+1)
    sizehint!(result3,N+1)
    S= T(0.0)
    for i in range(1,N)
        x1 = result1[end]
        y1 = result2[end]
        z1 = result3[end]
        S+=s
        x_new = x1 + s*f1(x1,y1,z1,S)
        push!(result1,x_new)
        y_new = y1 + s*f2(x1,y1,z1,S)
        push!(result2,y_new)
        z_new = z1 + s*f3(x1,y1,z1,S)
        push!(result3,z_new)
    end
    return result1, result2, result3
end

#ルンゲ・クッタ法
function rk4_1(f1,x0,stepsize,T,N)
    s = T(stepsize)
    result1 = T[x0]
    sizehint!(result1,N+1)
    S= T(0.0)
    for i in range(1,N)
        x1 = result1[end]
        S+=s
        k1 = f1(x1,S)
        k2 = f1(x1+k1*(s/2),S+(s/2))
        k3 = f1(x1+k2*(s/2),S+(s/2))
        k4 = f1(x1+(k3*s),S+s)
        x_new = x1 + (s)*(k1/6 + k2/3 + k3/3 + k4/6)
        push!(result1,x_new)
    end
    return result1
end

function rk4_2(f1,f2,x0,y0,stepsize,T,N)
    s = T(stepsize)
    result1 = [x0]
    result2 = [y0]
    sizehint!(result1,N+1)
    sizehint!(result2,N+1)
    S= T(0.0)
    for i in range(1,N)
        x1 = result1[end]
        y1 = result2[end]
        S+=s
        k1 = T(f1(x1,y1,S))
        j1 = T(f2(x1,y1,S))
        k2 = T(f1(x1+k1*(s/2),y1+j1*(s/2),S+(s/2)))
        j2 = T(f2(x1+k1*(s/2),y1+j1*(s/2),S+(s/2)))
        k3 = T(f1(x1+k2*(s/2),y1+j2*(s/2),S+(s/2)))
        j3 = T(f2(x1+k2*(s/2),y1+j2*(s/2),S+(s/2)))
        k4 = T(f1(x1+k3*s,y1+j3*s,S+s))
        j4 = T(f2(x1+k3*s,y1+j3*s,S+s))
        x_new = x1 + (s)*(k1/6 + k2/3 + k3/3 + k4/6)
        push!(result1,x_new)
        y_new = y1 + (s)*(j1/6 + j2/3 + j3/3 + j4/6)
        push!(result2,y_new)
    end
    return result1, result2
end

function rk4_3(f1,f2,f3,x0,y0,z0,stepsize,T,N)
    s = T(stepsize)
    result1 = T[x0]
    result2 = T[y0]
    result3 = T[z0]
    sizehint!(result1,N+1)
    sizehint!(result2,N+1)
    sizehint!(result3,N+1)
    S= T(0.0)
    for i in range(1,N)
        x1 = result1[end]
        y1 = result2[end]
        z1 = result3[end]
        S+=s
        k1 = f1(x1,y1,z1,S)
        j1 = f2(x1,y1,z1,S)
        l1 = f3(x1,y1,z1,S)
        k2 = f1(x1+k1*(s/2),y1+j1*(s/2),z1+l1*(s/2),S+(s/2))
        j2 = f2(x1+k1*(s/2),y1+j1*(s/2),z1+l1*(s/2),S+(s/2))
        l2 = f3(x1+k1*(s/2),y1+j1*(s/2),z1+l1*(s/2),S+(s/2))
        k3 = f1(x1+k2*(s/2),y1+j2*(s/2),z1+l2*(s/2),S+(s/2))
        j3 = f2(x1+k2*(s/2),y1+j2*(s/2),z1+l2*(s/2),S+(s/2))
        l3 = f3(x1+k2*(s/2),y1+j2*(s/2),z1+l2*(s/2),S+(s/2))
        k4 = f1(x1+k3*s,y1+j3*s,z1+l3*s,S+s)
        j4 = f2(x1+k3*s,y1+j3*s,z1+l3*s,S+s)
        l4 = f3(x1+k3*s,y1+k3*s,z1+l3*s,S+s)
        x_new = x1 + (s)*(k1/6 + k2/3+ k3/3 + k4/6)
        push!(result1,x_new)
        y_new = y1 + (s)*(j1/6 +j2/3 + j3/3 + j4/6)
        push!(result2,y_new)
        z_new = z1 + (s)*(l1/6 + l2/3 + l3/3 + l4/6)
        push!(result3,z_new)
    end
    return result1, result2, result3
end
\end{lstlisting}

例としてLorenz方程式での数値実験をEuler法を用いて行うプログラムのコードを掲載する.

\begin{lstlisting}[caption=\text{Lorenz方程式の関数を定義},label=code_lorenz]{}
using CairoMakie, FixedPointNumbers, DifferentialEquations
include("../numerical_scheme.jl")
setprecision(BigFloat,256)
#Lorenz方程式
#=パラメータの設定
σ = 10.0
ρ = 28.0
β = 2.6640625
=#

#float64
function f_float64_1(x::Float64,y::Float64,z::Float64,t::Float64)
    σ = 10.0
    return σ*(y - x)
end
function f_float64_2(x::Float64,y::Float64,z::Float64,t::Float64)
    ρ = 28.0
    return x*(ρ - z) - y
end
function f_float64_3(x::Float64,y::Float64,z::Float64,t::Float64)
    β = 2.6640625
    return x*y - β*z
end

#64ビットの固定小数点数で実行
function f_Q11f52_1(x::Q11f52,y::Q11f52,z::Q11f52,t::Q11f52)
    σ_Q11f52 = Q11f52(10.0)
    return σ_Q11f52*(y - x)
end
function f_Q11f52_2(x::Q11f52,y::Q11f52,z::Q11f52,t::Q11f52)
    ρ_Q11f52 = Q11f52(28.0)
    return x*(ρ_Q11f52 - z) - y
end
function f_Q11f52_3(x::Q11f52,y::Q11f52,z::Q11f52,t::Q11f52)
    β_Q11f52 = Q11f52(2.6640625)
    return x*y - β_Q11f52*z
end

function f_Q9f54_1(x::Q9f54,y::Q9f54,z::Q9f54,t::Q9f54)
    σ_Q9f54 = Q9f54(10.0)
    return σ_Q9f54*(y - x)    
end
function f_Q9f54_2(x::Q9f54,y::Q9f54,z::Q9f54,t::Q9f54)
    ρ_Q9f54 = Q9f54(28.0)
    return x*(ρ_Q9f54 - z) - y
end
function f_Q9f54_3(x::Q9f54,y::Q9f54,z::Q9f54,t::Q9f54)
    β_Q9f54 = Q9f54(2.6640625)
    return x*y - β_Q9f54*z
end

#BigFloatで実行
function f_BigFloat_1(x::BigFloat,y::BigFloat,z::BigFloat,t::BigFloat)
    σ_BigFloat = BigFloat("10.0")
    return σ_BigFloat*(y - x)
end
function f_BigFloat_2(x::BigFloat,y::BigFloat,z::BigFloat,t::BigFloat)
    ρ_BigFloat = BigFloat("28.0")
    return x*(ρ_BigFloat - z) - y
end
function f_BigFloat_3(x::BigFloat,y::BigFloat,z::BigFloat,t::BigFloat)
    β_BigFloat = BigFloat("2.6640625")
    return x*y - β_BigFloat*z
end
\end{lstlisting}

\begin{lstlisting}[caption={誤差を計算}, label=code_error]{}
#=Euler法での誤差の比較=#
#刻み幅 Δt = 2^(-7)
#それぞれの数値型で実行
ans_big = euler_3(f_BigFloat_1,f_BigFloat_2,f_BigFloat_3,BigFloat("0.5"),BigFloat("0.5"),BigFloat("0.5"),BigFloat(2^(-7)),BigFloat,Int(2^14))
ans_fl64 = euler_3(f_float64_1,f_float64_2,f_float64_3,0.5,0.5,0.5,2^(-7),Float64,Int(2^14))
ans_Q11f52 = euler_3(f_Q11f52_1,f_Q11f52_2,f_Q11f52_3,Q11f52(0.5),Q11f52(0.5),Q11f52(0.5),2^(-7),Q11f52,Int(2^14))
ans_Q9f54 = euler_3(f_Q9f54_1,f_Q9f54_2,f_Q9f54_3,Q9f54(0.5),Q9f54(0.5),Q9f54(0.5),2^(-7),Q9f54,Int(2^14))
#誤差を計算
ernrom1 = err_norm_3d(ans_big[1],ans_big[2],ans_bih[3],ans_fl64[1],ans_fl64[2],ans_fl64[])
ernrom2 = err_norm_3d(ans_big[1],ans_big[2],ans_bih[3],ans_Q11f52[1],ans_Q11f52[2],ans_Q11f52[])
ernrom3 = err_norm_3d(ans_big[1],ans_big[2],ans_bih[3],ans_Q9f54[1],ans_Q9f54[2],ans_Q9f54[])
\end{lstlisting}